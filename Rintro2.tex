% Options for packages loaded elsewhere
\PassOptionsToPackage{unicode}{hyperref}
\PassOptionsToPackage{hyphens}{url}
%
\documentclass[
]{article}
\usepackage{amsmath,amssymb}
\usepackage{iftex}
\ifPDFTeX
  \usepackage[T1]{fontenc}
  \usepackage[utf8]{inputenc}
  \usepackage{textcomp} % provide euro and other symbols
\else % if luatex or xetex
  \usepackage{unicode-math} % this also loads fontspec
  \defaultfontfeatures{Scale=MatchLowercase}
  \defaultfontfeatures[\rmfamily]{Ligatures=TeX,Scale=1}
\fi
\usepackage{lmodern}
\ifPDFTeX\else
  % xetex/luatex font selection
\fi
% Use upquote if available, for straight quotes in verbatim environments
\IfFileExists{upquote.sty}{\usepackage{upquote}}{}
\IfFileExists{microtype.sty}{% use microtype if available
  \usepackage[]{microtype}
  \UseMicrotypeSet[protrusion]{basicmath} % disable protrusion for tt fonts
}{}
\makeatletter
\@ifundefined{KOMAClassName}{% if non-KOMA class
  \IfFileExists{parskip.sty}{%
    \usepackage{parskip}
  }{% else
    \setlength{\parindent}{0pt}
    \setlength{\parskip}{6pt plus 2pt minus 1pt}}
}{% if KOMA class
  \KOMAoptions{parskip=half}}
\makeatother
\usepackage{xcolor}
\usepackage[margin=1in]{geometry}
\usepackage{color}
\usepackage{fancyvrb}
\newcommand{\VerbBar}{|}
\newcommand{\VERB}{\Verb[commandchars=\\\{\}]}
\DefineVerbatimEnvironment{Highlighting}{Verbatim}{commandchars=\\\{\}}
% Add ',fontsize=\small' for more characters per line
\usepackage{framed}
\definecolor{shadecolor}{RGB}{248,248,248}
\newenvironment{Shaded}{\begin{snugshade}}{\end{snugshade}}
\newcommand{\AlertTok}[1]{\textcolor[rgb]{0.94,0.16,0.16}{#1}}
\newcommand{\AnnotationTok}[1]{\textcolor[rgb]{0.56,0.35,0.01}{\textbf{\textit{#1}}}}
\newcommand{\AttributeTok}[1]{\textcolor[rgb]{0.13,0.29,0.53}{#1}}
\newcommand{\BaseNTok}[1]{\textcolor[rgb]{0.00,0.00,0.81}{#1}}
\newcommand{\BuiltInTok}[1]{#1}
\newcommand{\CharTok}[1]{\textcolor[rgb]{0.31,0.60,0.02}{#1}}
\newcommand{\CommentTok}[1]{\textcolor[rgb]{0.56,0.35,0.01}{\textit{#1}}}
\newcommand{\CommentVarTok}[1]{\textcolor[rgb]{0.56,0.35,0.01}{\textbf{\textit{#1}}}}
\newcommand{\ConstantTok}[1]{\textcolor[rgb]{0.56,0.35,0.01}{#1}}
\newcommand{\ControlFlowTok}[1]{\textcolor[rgb]{0.13,0.29,0.53}{\textbf{#1}}}
\newcommand{\DataTypeTok}[1]{\textcolor[rgb]{0.13,0.29,0.53}{#1}}
\newcommand{\DecValTok}[1]{\textcolor[rgb]{0.00,0.00,0.81}{#1}}
\newcommand{\DocumentationTok}[1]{\textcolor[rgb]{0.56,0.35,0.01}{\textbf{\textit{#1}}}}
\newcommand{\ErrorTok}[1]{\textcolor[rgb]{0.64,0.00,0.00}{\textbf{#1}}}
\newcommand{\ExtensionTok}[1]{#1}
\newcommand{\FloatTok}[1]{\textcolor[rgb]{0.00,0.00,0.81}{#1}}
\newcommand{\FunctionTok}[1]{\textcolor[rgb]{0.13,0.29,0.53}{\textbf{#1}}}
\newcommand{\ImportTok}[1]{#1}
\newcommand{\InformationTok}[1]{\textcolor[rgb]{0.56,0.35,0.01}{\textbf{\textit{#1}}}}
\newcommand{\KeywordTok}[1]{\textcolor[rgb]{0.13,0.29,0.53}{\textbf{#1}}}
\newcommand{\NormalTok}[1]{#1}
\newcommand{\OperatorTok}[1]{\textcolor[rgb]{0.81,0.36,0.00}{\textbf{#1}}}
\newcommand{\OtherTok}[1]{\textcolor[rgb]{0.56,0.35,0.01}{#1}}
\newcommand{\PreprocessorTok}[1]{\textcolor[rgb]{0.56,0.35,0.01}{\textit{#1}}}
\newcommand{\RegionMarkerTok}[1]{#1}
\newcommand{\SpecialCharTok}[1]{\textcolor[rgb]{0.81,0.36,0.00}{\textbf{#1}}}
\newcommand{\SpecialStringTok}[1]{\textcolor[rgb]{0.31,0.60,0.02}{#1}}
\newcommand{\StringTok}[1]{\textcolor[rgb]{0.31,0.60,0.02}{#1}}
\newcommand{\VariableTok}[1]{\textcolor[rgb]{0.00,0.00,0.00}{#1}}
\newcommand{\VerbatimStringTok}[1]{\textcolor[rgb]{0.31,0.60,0.02}{#1}}
\newcommand{\WarningTok}[1]{\textcolor[rgb]{0.56,0.35,0.01}{\textbf{\textit{#1}}}}
\usepackage{graphicx}
\makeatletter
\def\maxwidth{\ifdim\Gin@nat@width>\linewidth\linewidth\else\Gin@nat@width\fi}
\def\maxheight{\ifdim\Gin@nat@height>\textheight\textheight\else\Gin@nat@height\fi}
\makeatother
% Scale images if necessary, so that they will not overflow the page
% margins by default, and it is still possible to overwrite the defaults
% using explicit options in \includegraphics[width, height, ...]{}
\setkeys{Gin}{width=\maxwidth,height=\maxheight,keepaspectratio}
% Set default figure placement to htbp
\makeatletter
\def\fps@figure{htbp}
\makeatother
\setlength{\emergencystretch}{3em} % prevent overfull lines
\providecommand{\tightlist}{%
  \setlength{\itemsep}{0pt}\setlength{\parskip}{0pt}}
\setcounter{secnumdepth}{-\maxdimen} % remove section numbering
\usepackage{hyperref}
\hypersetup{colorlinks=true, linkcolor=red, urlcolor=blue}
\ifLuaTeX
  \usepackage{selnolig}  % disable illegal ligatures
\fi
\IfFileExists{bookmark.sty}{\usepackage{bookmark}}{\usepackage{hyperref}}
\IfFileExists{xurl.sty}{\usepackage{xurl}}{} % add URL line breaks if available
\urlstyle{same}
\hypersetup{
  pdftitle={(Fast) Introduction to R - Class 2},
  pdfauthor={Joana Cima},
  hidelinks,
  pdfcreator={LaTeX via pandoc}}

\title{(Fast) Introduction to R - Class 2}
\usepackage{etoolbox}
\makeatletter
\providecommand{\subtitle}[1]{% add subtitle to \maketitle
  \apptocmd{\@title}{\par {\large #1 \par}}{}{}
}
\makeatother
\subtitle{Jump into a notebook}
\author{Joana Cima}
\date{13 outubro 2023}

\begin{document}
\maketitle

\hypertarget{my-beamer}{%
\section{My beamer}\label{my-beamer}}

\begin{quote}
BlaBlaBla
\end{quote}

\hypertarget{outline}{%
\section{Outline}\label{outline}}

\begin{enumerate}
\def\labelenumi{\arabic{enumi}.}
\tightlist
\item
  Motivation
\item
  Data
\item
  Conceptual discussion
\end{enumerate}

\hypertarget{import-data-from-an-excel-file}{%
\section{3. Import data (from an excel
file)}\label{import-data-from-an-excel-file}}

\begin{Shaded}
\begin{Highlighting}[]
\NormalTok{nlswork }\OtherTok{\textless{}{-}} \FunctionTok{as.data.frame}\NormalTok{(}\FunctionTok{read\_excel}\NormalTok{(}\StringTok{"nlswork.xlsx"}\NormalTok{))}
\end{Highlighting}
\end{Shaded}

\hypertarget{drop-missing-values}{%
\section{3.1 Drop missing values}\label{drop-missing-values}}

\begin{Shaded}
\begin{Highlighting}[]
\NormalTok{nlswork\_no\_na }\OtherTok{\textless{}{-}} \FunctionTok{drop\_na}\NormalTok{(nlswork)}
\end{Highlighting}
\end{Shaded}

\hypertarget{descriptive-statistics}{%
\section{4. Descriptive statistics}\label{descriptive-statistics}}

(\ldots)

\hypertarget{regression-analysis}{%
\section{5. Regression analysis}\label{regression-analysis}}

\hypertarget{regression-analysis-ols}{%
\subsection{5.1 Regression analysis:
OLS}\label{regression-analysis-ols}}

\hypertarget{variable-selection}{%
\subsubsection{5.1.1. Variable Selection}\label{variable-selection}}

Selecting appropriate variables for our model is critical to derive
accurate and meaningful results.

\begin{verbatim}
##           Overall
## age      11.92846
## collgrad 41.15039
## union    25.20145
\end{verbatim}

\hypertarget{our-regression-table}{%
\subsubsection{5.1.2. Our regression Table}\label{our-regression-table}}

Exemplo de texto.

\begin{table}[!ht] \centering 
  \caption{Regression analysis} 
  \label{regressions} 
\begin{tabular}{@{\extracolsep{5pt}}lcc} 
\\[-1.8ex]\hline 
\hline \\[-1.8ex] 
 & Model (1) & Model (2) \\ 
 age & 0.010$^{***}$ & 0.007$^{***}$ \\ 
  & (0.001) & (0.001) \\ 
  collgrad &  & 0.383$^{***}$ \\ 
  &  & (0.009) \\ 
  union &  & 0.217$^{***}$ \\ 
  &  & (0.009) \\ 
 \textit{N} & 13,452 & 13,452 \\ 
R$^{2}$ & 0.021 & 0.175 \\ 
\hline 
\hline \\[-1.8ex] 
\textit{Notes:} & \multicolumn{2}{l}{$^{***}$Significant at the 1 percent level.} \\ 
 & \multicolumn{2}{l}{$^{**}$Significant at the 5 percent level.} \\ 
 & \multicolumn{2}{l}{$^{*}$Significant at the 10 percent level.} \\ 
 & \multicolumn{2}{l}{Standard errors in parentheses.} \\ 
\end{tabular} 
\end{table}

\hypertarget{hypothesis-testing-automatic-command}{%
\subsubsection{5.1.3. HYPOTHESIS TESTING: automatic
command}\label{hypothesis-testing-automatic-command}}

Next, we will be testing specific hypotheses about the coefficients in
our regression model.

\begin{enumerate}
\def\labelenumi{\arabic{enumi}.}
\item
  \textbf{Testing if the Coefficient for \texttt{age} is Zero:}\\
  We aim to test the null hypothesis that: \begin{equation}
  H_0: \beta_{\text{age}} = 0
  \end{equation} This tests if \texttt{age} has any influence on the
  dependent variable, after adjusting for other variables in the model.
\item
  \textbf{Testing if the Coefficients for \texttt{union} and
  \texttt{collgrad} are Equal:}\\
  We will test the null hypothesis that: \begin{equation}
  H_0: \beta_{\text{union}} = \beta_{\text{collgrad}}
  \end{equation} This checks if the effect of being in a union on the
  dependent variable is the same as the effect of being a college
  graduate, when other factors are held constant.
\end{enumerate}

\begin{verbatim}
## Linear hypothesis test
## 
## Hypothesis:
## age = 0
## 
## Model 1: restricted model
## Model 2: ln_wage ~ age + collgrad + union
## 
##   Res.Df    RSS Df Sum of Sq      F    Pr(>F)    
## 1  13449 2368.0                                  
## 2  13448 2343.2  1    24.793 142.29 < 2.2e-16 ***
## ---
## Signif. codes:  0 '***' 0.001 '**' 0.01 '*' 0.05 '.' 0.1 ' ' 1
\end{verbatim}

\begin{verbatim}
## Linear hypothesis test
## 
## Hypothesis:
## - collgrad  + union = 0
## 
## Model 1: restricted model
## Model 2: ln_wage ~ age + collgrad + union
## 
##   Res.Df    RSS Df Sum of Sq      F    Pr(>F)    
## 1  13449 2371.0                                  
## 2  13448 2343.2  1    27.743 159.22 < 2.2e-16 ***
## ---
## Signif. codes:  0 '***' 0.001 '**' 0.01 '*' 0.05 '.' 0.1 ' ' 1
\end{verbatim}

\hypertarget{additional-quality-measures-aic-bic}{%
\subsubsection{5.1.4. Additional quality measures: AIC \&
BIC}\label{additional-quality-measures-aic-bic}}

In both the AIC and BIC criteria, the model with the lower values is
favored as it suggests a better balance between model fit and model
complexity.

\begin{verbatim}
## [1] 16981.05
\end{verbatim}

\begin{verbatim}
## [1] 17003.57
\end{verbatim}

\begin{verbatim}
## [1] 14676.36
\end{verbatim}

\begin{verbatim}
## [1] 14713.9
\end{verbatim}

\hypertarget{colinearity-vif}{%
\subsubsection{5.1.5. COLINEARITY: VIF}\label{colinearity-vif}}

The Variance Inflation Factor (VIF) assesses the severity of
multicollinearity in a regression, with values greater than 10
suggesting high correlation between predictors (Wooldrige; Verbeek).

\begin{verbatim}
##      age collgrad    union 
## 1.018819 1.025643 1.007647
\end{verbatim}

\hypertarget{heteroskedasticity}{%
\subsubsection{5.1.6. HETEROSKEDASTICITY}\label{heteroskedasticity}}

Homoscedasticity is a fundamental assumption underlying standard linear
regression models, stipulating that the variance of the residuals
remains constant across levels of the independent variables. This
property ensures that the ordinary least squares (OLS) estimator remains
the best linear unbiased estimator (BLUE), providing minimum variance.
Violations of homoscedasticity, known as heteroscedasticity, can lead to
inefficient and potentially biased coefficient estimates, as well as
unreliable standard errors. To rigorously assess the presence of
homoscedasticity, researchers often employ diagnostic tests, such as the
Breusch-Pagan test.

\hypertarget{breusch-pagan-test}{%
\paragraph{Breusch-Pagan test}\label{breusch-pagan-test}}

\begin{verbatim}
## 
##  studentized Breusch-Pagan test
## 
## data:  ols2
## BP = 220.76, df = 3, p-value < 2.2e-16
\end{verbatim}

The results from the Breusch-Pagan test for the \texttt{ols2} model
suggest the presence of heteroscedasticity. The test statistic is
\(BP = 199.54\) with a degree of freedom (df) of 4. The low p-value,
effectively zero at \(p < 2.2e-16\), leads us to reject the null
hypothesis of homoscedasticity.

\hypertarget{robust-estimation}{%
\subsubsection{5.1.7. Robust estimation}\label{robust-estimation}}

t test of coefficients:

\begin{verbatim}
          Estimate Std. Error t value  Pr(>|t|)    
\end{verbatim}

(Intercept) 1.38759270 0.01699578 81.643 \textless{} 2.2e-16
\textbf{\emph{ age 0.00675636 0.00057968 11.655 \textless{} 2.2e-16 }}
collgrad 0.38330049 0.00950024 40.346 \textless{} 2.2e-16 \textbf{\emph{
union 0.21681639 0.00828751 26.162 \textless{} 2.2e-16 }} --- Signif.
codes: 0 `\emph{\textbf{' 0.001 '}' 0.01 '}' 0.05 `.' 0.1 ' ' 1

\% Table created by stargazer v.5.2.3 by Marek Hlavac, Social Policy
Institute. E-mail: marek.hlavac at gmail.com \% Date and time: sex, out
13, 2023 - 10:38:15

\begin{table}[!htbp] \centering 
  \caption{} 
  \label{} 
\begin{tabular}{@{\extracolsep{5pt}}lc} 
\\[-1.8ex]\hline 
\hline \\[-1.8ex] 
\\[-1.8ex] &   \\ 
\hline \\[-1.8ex] 
 age & 0.0068$^{***}$ \\ 
  & (0.0006) \\ 
  & \\ 
 collgrad & 0.3833$^{***}$ \\ 
  & (0.0095) \\ 
  & \\ 
 union & 0.2168$^{***}$ \\ 
  & (0.0083) \\ 
  & \\ 
 Constant & 1.3876$^{***}$ \\ 
  & (0.0170) \\ 
  & \\ 
\hline 
\hline \\[-1.8ex] 
\textit{Notes:} & \multicolumn{1}{l}{$^{***}$Significant at the 1 percent level.} \\ 
 & \multicolumn{1}{l}{$^{**}$Significant at the 5 percent level.} \\ 
 & \multicolumn{1}{l}{$^{*}$Significant at the 10 percent level.} \\ 
\end{tabular} 
\end{table}

\hypertarget{coefficient-interpretation-with-a-full-model-model-2}{%
\subsubsection{5.1.8. Coefficient interpretation with a full model
(Model 2)}\label{coefficient-interpretation-with-a-full-model-model-2}}

Given that our dependent variable is the natural logarithm of the
salary, the coefficients in this regression model represent percentage
changes in the salary for a one-unit change in the independent
variables.For every additional year in age, the salary is expected to
increase by approximately 0.6779236\%, holding all other factors
constant. Being a college graduate is associated with an estimated
increase of about 46.7118818\% in the salary, compared to not being a
college graduate. Being a member of a union is associated with an
approximate 24.2116016\% increase in salary, holding everything else
constant. This effect is highly significant (p-value \textless{} 2e-16).

\hypertarget{binary-choice-models}{%
\subsection{5.2. Binary choice models}\label{binary-choice-models}}

When the dependent variable is binary, ordinary least squares regression
is not suitable due to the non-continuous nature of the outcome
variable. Models such as Logit and Probit are specifically designed to
handle such binary outcomes. In our analysis, where we aim to understand
the probability of a worker being unionized, these models are
appropriate choices as they provide insights into the factors
influencing this binary decision.These models show the direction of the
relationship between independent variables and the dependent variable
but don't quantify the magnitude.

\begin{verbatim}
## 
## Call:
## glm(formula = union ~ ., family = binomial(link = "probit"), 
##     data = db_ols)
## 
## Coefficients:
##              Estimate Std. Error z value Pr(>|z|)    
## (Intercept) -1.963478   0.073502 -26.713   <2e-16 ***
## age         -0.002860   0.001959  -1.459    0.144    
## ln_wage      0.738308   0.030169  24.472   <2e-16 ***
## collgrad    -0.001693   0.032350  -0.052    0.958    
## ---
## Signif. codes:  0 '***' 0.001 '**' 0.01 '*' 0.05 '.' 0.1 ' ' 1
## 
## (Dispersion parameter for binomial family taken to be 1)
## 
##     Null deviance: 14463  on 13451  degrees of freedom
## Residual deviance: 13733  on 13448  degrees of freedom
## AIC: 13741
## 
## Number of Fisher Scoring iterations: 4
\end{verbatim}

\begin{table}[!ht] \centering 
  \caption{Regression} 
  \label{regressions} 
\begin{tabular}{@{\extracolsep{5pt}}lcc} 
\\[-1.8ex]\hline 
\hline \\[-1.8ex] 
 &  Probit & Logit \\ 
 age & $-$0.0029 & $-$0.0055 \\ 
  & (0.0020) & (0.0034) \\ 
  ln\_wage & 0.7383$^{***}$ & 1.2492$^{***}$ \\ 
  & (0.0302) & (0.0524) \\ 
  collgrad & $-$0.0017 & $-$0.0025 \\ 
  & (0.0323) & (0.0545) \\ 
 \textit{N} & 13,452 & 13,452 \\ 
Log Likelihood & $-$6,866.5360 & $-$6,873.1680 \\ 
Akaike Inf. Crit. & 13,741.0700 & 13,754.3400 \\ 
\hline 
\hline \\[-1.8ex] 
\textit{Notes:} & \multicolumn{2}{l}{$^{***}$Significant at the 1 percent level.} \\ 
 & \multicolumn{2}{l}{$^{**}$Significant at the 5 percent level.} \\ 
 & \multicolumn{2}{l}{$^{*}$Significant at the 10 percent level.} \\ 
 & \multicolumn{2}{l}{Standard errors in parentheses.} \\ 
\end{tabular} 
\end{table}

\hypertarget{marginal-effects}{%
\subsubsection{5.2.1. Marginal effects}\label{marginal-effects}}

Marginal effects are crucial in non-linear models like Logit and Probit.
While the model's coefficients tell us about the direction of effects,
they don't show the actual change in probability for a one-unit change
in the predictor. Marginal effects provide this information, making it
easier to understand the real-world impact of each variable on the
outcome.

\hypertarget{marginal-effects---probit}{%
\paragraph{5.2.2. Marginal effects -
probit}\label{marginal-effects---probit}}

\begin{verbatim}
##    factor     AME     SE       z      p   lower  upper
##       age -0.0008 0.0006 -1.4596 0.1444 -0.0019 0.0003
##  collgrad -0.0005 0.0093 -0.0523 0.9583 -0.0186 0.0177
##   ln_wage  0.2114 0.0082 25.8419 0.0000  0.1954 0.2274
\end{verbatim}

\hypertarget{marginal-effects---logit}{%
\paragraph{5.2.3. Marginal effects -
logit}\label{marginal-effects---logit}}

\begin{verbatim}
##    factor     AME     SE       z      p   lower  upper
##       age -0.0009 0.0006 -1.6289 0.1033 -0.0020 0.0002
##  collgrad -0.0004 0.0091 -0.0458 0.9635 -0.0182 0.0174
##   ln_wage  0.2082 0.0082 25.2995 0.0000  0.1921 0.2244
\end{verbatim}

\hypertarget{summary-of-marginal-effects-for-probit-model-on-union-membership}{%
\paragraph{5.2.4. Summary of Marginal Effects for Probit Model on Union
Membership:}\label{summary-of-marginal-effects-for-probit-model-on-union-membership}}

\begin{itemize}
\item
  \textbf{Age:} A one-year increase in age is linked to a decrease in
  the likelihood of a worker being unionized by 0.08 percentage points.
  This effect, though statistically significant, is minimal, indicating
  age may have only a slight influence on union membership.
\item
  \textbf{College Graduation (collgrad):} Being a college graduate
  reduces the probability of being in a union by 0.05 percentage points.
  The small magnitude of this effect suggests that in this context,
  educational attainment (specifically having a college degree) has
  limited influence on union membership.
\item
  \textbf{Log of Wage (ln\_wage):} A one-unit increase in the logarithm
  of wage is associated with a 21.14 percentage point increase in the
  likelihood of a worker being unionized. This significant effect
  underscores the importance of wage in a worker's decision to join a
  union.
\end{itemize}

\hypertarget{machine-learning-linear-regression-vs.-random-forest}{%
\section{6. Machine Learning: Linear Regression vs.~Random
Forest}\label{machine-learning-linear-regression-vs.-random-forest}}

In this section, we will train both a traditional linear regression
model and a Random Forest model to predict \texttt{ln\_wage} and then
compare their performance on a test dataset.

\hypertarget{splitting-data-into-training-and-testing-sets}{%
\subsection{6.1. Splitting Data into Training and Testing
Sets}\label{splitting-data-into-training-and-testing-sets}}

Before we train our models, we need to partition our dataset into a
training set, used to train the models, and a test set, used to evaluate
their performance.

\hypertarget{training-and-evaluating-linear-regression}{%
\subsection{6.2. Training and Evaluating Linear
Regression}\label{training-and-evaluating-linear-regression}}

\begin{verbatim}
## [1] 0.3517657
\end{verbatim}

\hypertarget{training-and-evaluating-random-forest}{%
\subsection{6.3. Training and Evaluating Random
Forest}\label{training-and-evaluating-random-forest}}

\begin{verbatim}
## [1] 0.3137955
\end{verbatim}

\hypertarget{comparing-model-performances}{%
\subsection{6.4. Comparing Model
Performances}\label{comparing-model-performances}}

With the results in hand, let's compare the performance of the two
models. RMSE provides a measure of the magnitude of the prediction
errors. Lower values of RMSE indicate a better fit of the model to the
data.

\begin{verbatim}
##               Model      RMSE
## 1 Linear Regression 0.3517657
## 2     Random Forest 0.3137955
\end{verbatim}

\hypertarget{assessment}{%
\section{7. Assessment}\label{assessment}}

\hypertarget{problem-1-data-importing}{%
\subsection{Problem 1: Data Importing}\label{problem-1-data-importing}}

Import the ``card'' dataset.

\begin{Shaded}
\begin{Highlighting}[]
\NormalTok{card}\OtherTok{\textless{}{-}}\FunctionTok{as.data.frame}\NormalTok{(}\FunctionTok{read\_excel}\NormalTok{(}\StringTok{"card.xlsx"}\NormalTok{))}
\end{Highlighting}
\end{Shaded}

\hypertarget{problem-2-drop-the-missing-data}{%
\subsection{Problem 2: Drop the missing
data}\label{problem-2-drop-the-missing-data}}

\begin{Shaded}
\begin{Highlighting}[]
\CommentTok{\#}\RegionMarkerTok{BEGIN}\CommentTok{ SOLUTION}
\NormalTok{card\_no\_na}\OtherTok{\textless{}{-}}\FunctionTok{drop\_na}\NormalTok{(card)}
\CommentTok{\#}\RegionMarkerTok{END}\CommentTok{ SOLUTION}
\end{Highlighting}
\end{Shaded}

\hypertarget{problem-3}{%
\subsection{Problem 3}\label{problem-3}}

Estimate a linear regression model that uses the log of the salary as
the dependent variable and IQ, married, age, and educ as independent
variables.

\begin{Shaded}
\begin{Highlighting}[]
\CommentTok{\#}\RegionMarkerTok{BEGIN}\CommentTok{ SOLUTION}


\CommentTok{\#}\RegionMarkerTok{END}\CommentTok{ SOLUTION}
\end{Highlighting}
\end{Shaded}

\hypertarget{problem-4}{%
\subsection{Problem 4:}\label{problem-4}}

Which variables are the most important in the model? Explain

\begin{Shaded}
\begin{Highlighting}[]
\CommentTok{\#}\RegionMarkerTok{BEGIN}\CommentTok{ SOLUTION}


\CommentTok{\#}\RegionMarkerTok{END}\CommentTok{ SOLUTION}
\end{Highlighting}
\end{Shaded}

\hypertarget{problem-5-what-can-you-conclude-regarding-homoscedasticity}{%
\subsection{Problem 5: What can you conclude regarding
homoscedasticity?}\label{problem-5-what-can-you-conclude-regarding-homoscedasticity}}

\begin{Shaded}
\begin{Highlighting}[]
\CommentTok{\#}\RegionMarkerTok{BEGIN}\CommentTok{ SOLUTION}


\CommentTok{\#}\RegionMarkerTok{END}\CommentTok{ SOLUTION}
\end{Highlighting}
\end{Shaded}

\hypertarget{problem-6}{%
\subsection{Problem 6:}\label{problem-6}}

Interpret the coefficients of the variables

\begin{Shaded}
\begin{Highlighting}[]
\CommentTok{\#}\RegionMarkerTok{BEGIN}\CommentTok{ SOLUTION}

\CommentTok{\#}\RegionMarkerTok{END}\CommentTok{ SOLUTION}
\end{Highlighting}
\end{Shaded}

\hypertarget{problem-7}{%
\subsection{Problem 7:}\label{problem-7}}

Create a binary variable that takes the value 1 if the salary is above
the average and 0 otherwise

\begin{Shaded}
\begin{Highlighting}[]
\NormalTok{average\_wage }\OtherTok{\textless{}{-}} \FunctionTok{mean}\NormalTok{(card\_no\_na}\SpecialCharTok{$}\NormalTok{wage)}

\CommentTok{\# Create a binary variable (high wage): 1 if wage is above average, 0 otherwise}
\NormalTok{card\_no\_na }\OtherTok{\textless{}{-}}\NormalTok{ card\_no\_na }\SpecialCharTok{\%\textgreater{}\%}
  \FunctionTok{mutate}\NormalTok{(}\AttributeTok{high\_wage =} \FunctionTok{ifelse}\NormalTok{(wage }\SpecialCharTok{\textgreater{}}\NormalTok{ average\_wage, }\DecValTok{1}\NormalTok{, }\DecValTok{0}\NormalTok{))}
\end{Highlighting}
\end{Shaded}

\hypertarget{problem-8}{%
\subsection{Problem 8:}\label{problem-8}}

Estimate a logit model to explain the probability of an individual
having a salary above the average, using the same independent variables
as in the linear regression mode.

\begin{Shaded}
\begin{Highlighting}[]
\CommentTok{\#}\RegionMarkerTok{BEGIN}\CommentTok{ SOLUTION}

\CommentTok{\#}\RegionMarkerTok{END}\CommentTok{ SOLUTION}
\end{Highlighting}
\end{Shaded}

\hypertarget{problem-9}{%
\subsection{Problem 9:}\label{problem-9}}

Discuss how the independent variables are positively/negatively related
to the probability of the salary being above the average.

\begin{Shaded}
\begin{Highlighting}[]
\CommentTok{\#}\RegionMarkerTok{BEGIN}\CommentTok{ SOLUTION}

\CommentTok{\#}\RegionMarkerTok{END}\CommentTok{ SOLUTION}
\end{Highlighting}
\end{Shaded}


\end{document}
